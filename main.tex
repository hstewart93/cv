%%%%%%%%%%%%%%%%%%%%%%%%%%%%%%%%%%%%%%%%%
% Developer CV
% LaTeX Class
% Version 2.0 (12/10/23)
%
% This class originates from:
% http://www.LaTeXTemplates.com
%
% Authors:
% Omar Roldan
% Based on a template by  Jan Vorisek (jan@vorisek.me)
% Based on a template by Jan Küster (info@jankuester.com)
% Modified for LaTeX Templates by Vel (vel@LaTeXTemplates.com)
%
% License:
% The MIT License (see included LICENSE file)
%
%%%%%%%%%%%%%%%%%%%%%%%%%%%%%%%%%%%%%%%%%

%----------------------------------------------------------------------------------------
%	PACKAGES AND OTHER DOCUMENT CONFIGURATIONS
%----------------------------------------------------------------------------------------

\documentclass[9pt]{developercv} % Default font size, values from 8-12pt are recommended
\usepackage{multicol}
% \usepackage[]{fancyhdr}
\setlength{\columnsep}{0mm}
%----------------------------------------------------------------------------------------
\usepackage{lipsum}  


\begin{document}
% \pagestyle{fancy}
% \fancyfoot{}
%----------------------------------------------------------------------------------------
%	TITLE AND CONTACT INFORMATION
%----------------------------------------------------------------------------------------

\begin{minipage}[t]{0.5\textwidth} 
	\vspace{-\baselineskip} % Required for vertically aligning minipages
	
	{ \fontsize{16}{20} \textcolor{black}{\textbf{\MakeUppercase{Hattie Stewart}}}} % First name
	
	\vspace{6pt}
	
	% {\Large Developer $\sim$ Engineer} % Career or current job title
\end{minipage}
\hfill
\begin{minipage}[t]{0.2\textwidth} % 20% of the page width for the first row of icons
	\vspace{-\baselineskip} % Required for vertically aligning minipages
	
	% The first parameter is the FontAwesome icon name, the second is the box size and the third is the text
	\icon{Globe}{11}{\href{https://hstewart93.github.io}{hattie-stewart}}\\ 
    % \icon{Phone}{11}{07724 747 385}\\
    \icon{MapMarker}{11}{Bristol, UK}\\
	
\end{minipage}
\begin{minipage}[t]{0.27\textwidth} % 27% of the page width for the second row of icons
	\vspace{-\baselineskip} % Required for vertically aligning minipages
	
	% \icon{Envelope}{11}{\href{mailto:email@example.com}{harriet.stt@gmail.com}}\\	
    \icon{Github}{11}{\href{https://github.com/hstewart93}{github.com/hstewart93}}\\
    \icon{LinkedinSquare}{11}{\href{https://www.linkedin.com/in/hattie-stewart-7400b5162/}{/in/hattie-stewart}}\\    
    
\end{minipage}


%----------------------------------------------------------------------------------------
%	INTRODUCTION, SKILLS AND TECHNOLOGIES
%----------------------------------------------------------------------------------------

\begin{minipage}[t]{0.46\textwidth}
    % \vspace{-40pt}
    \cvsect{Summary}
	\vspace{-6pt}
 
    %Dummy text
    I am a PhD candidate in ML for Astrophysics with a background in software engineering, and have submitted my thesis for defense. I am highly committed to software best practices. I am particularly interested in the use of unsupervised methods to extract information from large data sets without the need of contextual information; using the power of ML to let the data speak for itself. \\
\end{minipage}
\hfill % Whitespace between
\begin{minipage}[t]{0.465\textwidth}
    \cvsect{Proficient Skills}
    \vspace{-6pt}
    
    \begin{minipage}[t]{0.2\textwidth}
        \textbf{Languages:}
    \end{minipage}
    \hfill
    \begin{minipage}[t]{0.73\textwidth}
      Python, SQL
    \end{minipage}
    \vspace{4mm}
    
    \begin{minipage}[t]{0.2\textwidth}
        \textbf{Technologies:}
    \end{minipage}
    \hfill
    \begin{minipage}[t]{0.73\textwidth}
      Git, GitHub, TensorFlow, PyTorch, Keras, SciPy, NumPy, Pandas, Scikit-learn, Jupyter, Matplotlib, Django, CI, CD, TDD, venv, LaTeX, JIRA, Trello, Agile, Docker, REST API, OpenAPI
    \end{minipage}
    
\end{minipage}

%----------------------------------------------------------------------------------------
%	EXPERIENCE
%----------------------------------------------------------------------------------------
% \vspace{-10 pt}
\cvsect{Professional Experience}
\begin{entrylist}
    \entry
		{2022}
		{Industry Placement - SciML group}
		{Rutherford Appleton Laboratory}
  		{\vspace{-10pt}
        \begin{itemize}[noitemsep,topsep=0pt,parsep=0pt,partopsep=0pt, leftmargin=-1pt]
            \item Investigating VAEs and AEs applied to scientific data sets for image segmentation.
        \end{itemize}
        \texttt{TensorFlow} \slashsep \texttt{Keras} \slashsep \texttt{PyTorch} \slashsep \texttt{Numpy} \slashsep \texttt{Pandas}} 
    \entry
        {2019 - 2022}
		{Freelance Software Engineer}
		{Uncommon Correlation}
		{\vspace{-10pt}
        \begin{itemize}[noitemsep,topsep=0pt,parsep=0pt,partopsep=0pt, leftmargin=-1pt]
            \item Python and Django back-end development for data governance and secure data processing systems.
        \end{itemize}}
	\entry
        {2018 - 2019}
		{Software Engineer}
		{Omni Digital}
		{\vspace{-10pt}
        \begin{itemize}[noitemsep,topsep=0pt,parsep=0pt,partopsep=0pt, leftmargin=-1pt]
            \item Back-end development,  building robust systems in Python and Django. Experience creating and implementing an open API specification. OOP, using modern tools/technologies.
            \item Front-end engineering and infrastructure experience.
            \item Committed to best practices, including comprehensive unit and function testing, as well as smoke tests and peer review.
            \item Experience in leading and working under Agile project management method.
            \item Project lead on, and assisted in pitching for, Google funded project; involving assisting in project and team management, making technical decisions and client liaison.
        \end{itemize} 
        \texttt{Python} \slashsep \texttt{TDD} \slashsep \texttt{pytest} \slashsep \texttt{unittest} \slashsep \texttt{Travis CI} \slashsep \texttt{CD} \slashsep \texttt{Django} \slashsep \texttt{Wagtail} \slashsep \texttt{PostgreSQL} \slashsep \texttt{HTML} \slashsep \texttt{Django Templates} \slashsep \texttt{CSS} \slashsep \texttt{Stylus} \slashsep \texttt{Ansible} \slashsep \texttt{Docker} \slashsep \texttt{Terraform} \slashsep \texttt{Digital Ocean} \slashsep \texttt{JIRA} \slashsep \texttt{REST API} \slashsep \texttt{OpenAPI}}
	\entry
		{2017 - 2018}
		{Programmer}
		{WebTMS}
		{\vspace{-10pt}
        \begin{itemize}[noitemsep,topsep=0pt,parsep=0pt,partopsep=0pt, leftmargin=-1pt]
            \item Continuous development of the WebTMS system, both back-end and front-end development. 
            \item Data migration and database administration.
            \item Providing first line technical support and training to clients and agents.
        \end{itemize} 
        \texttt{T-SQL} \slashsep \texttt{JavaScript} \slashsep \texttt{ASP.Net} \slashsep \texttt{VB.NET} \slashsep \texttt{JQuery} \slashsep \texttt{HTML} \slashsep \texttt{CSS} \slashsep \texttt{Visual Studio} \slashsep \texttt{IIS} \slashsep \texttt{TFS}}

\end{entrylist}


%----------------------------------------------------------------------------------------
%	EDUCATION
%----------------------------------------------------------------------------------------
\vspace{-10 pt}
\cvsect{Education}
\begin{entrylist}
    \entry
		{2019 - present}
		{PhD Candidate}
		{University of Bristol}
		{UKRI Centre for Doctoral Training in Artificial                                Intelligence, Machine Learning \& Advanced Computing \\
            \textnormal{\textit{Mark Birkinshaw, Natasha Maddox \& Ben Maughan}}
        % \vspace{-10pt}
        \begin{itemize}[noitemsep,topsep=0pt,parsep=0pt,partopsep=0pt, leftmargin=-1pt]
            \item PhD Project: Using ML image segmentation methods to extract astrophysical objects and their information from next generation radio survey data, in particular the Square Kilometre Array (SKA). Development of a radio image segmentation pipeline for extremely large data sets, ContinUNet, using U-Net as segmentation method.
            \item Experience in Bayesian statistics, data science techniques, computer vision and image segmentation tasks.
        \end{itemize} 
        \texttt{Python} \slashsep \texttt{R} \slashsep \texttt{NumPy} \slashsep \texttt{Pandas} \slashsep \texttt{SciPy} \slashsep \texttt{OpenCV} \slashsep \texttt{Scikit-learn} \slashsep \texttt{TensorFlow} \slashsep \texttt{Keras} \slashsep \texttt{PyTorch} \slashsep \texttt{Docker} \slashsep \texttt{DockerHub} \slashsep \texttt{HPC} \slashsep \texttt{CNNs} \slashsep \texttt{AEs} \slashsep \texttt{VAEs} \slashsep \texttt{Matplotlib} \slashsep \texttt{Seaborn} \slashsep \texttt{Git} \slashsep \texttt{pipenv} \slashsep \texttt{Linux}}
  %   \entry
		% {2022}
		% {Industry Placement - SciML group}
		% {Rutherford Appleton Laboratory}
		% {Investigating VAEs and AEs applied to scientific data sets for image segmentation.}
    \entry
		{2012 - 2016}
		{MPhys Physics with Astrophysics 2.1 (hons)}
		{The Unversity of Manchester}
		{MPhys project: Design and development of an extensive software system in         Python to automate the operation of a remote control telescope, including         reduction and analysis of collected data.}
    \entry
		{2010 - 2012}
		{A Levels}
		{The Thomas Hardye School}
		{A-levels: A*AA in Mathematics, Physics and Chemistry respectively.\\
            AS-levels: A in Music and B in Further Mathematics. Extended project: A*
            }
	\entry
		{2007 - 2010}
		{GCSEs}
		{The Purbeck School}
		{10 GCSEs, including 5 A*s and 5As with Science, Mathematics and English to       A* grade.}
\end{entrylist}


%----------------------------------------------------------------------------------------
%	PUBLICATIONS
%----------------------------------------------------------------------------------------
% \vspace{-10 pt}
\cvsect{Publications}
\begin{entrylist}
    \entry
		{Submitted}
		{\href{https://hstewart93.github.io/files/continunet_rasti_29_02_2024.pdf}{ContinUNet: Deep Radio Image Segmentation in the SKA Era with U-Net}}
		{Journal}
		{\textnormal{\textit{Royal Astronomical Society Techniques \& Instruments         (RASTI)}} \\
            Hattie Stewart, Mark Birkinshaw, Jason Yeung, Natasha Maddox, Ben Maughan,    Jeyan Thiyagalingam}
    \entry
		{Published}
		{\href{https://doi.org/10.1007/978-3-031-34167-0_28}{Radio Image Segmentation with Autoencoders}}
		{Conference Proceedings}
		{\textnormal{\textit{Proceedings of the ML4Astro International Conference}} \\Hattie Stewart, Jason Yeung, Mark Birkinshaw}
	\entry
		{Accepted}
		{\href{https://hstewart93.github.io/files/Proceedings_of_the_International_Astronomical_Union_Deep_Radio_Image_Segmentation.pdf}{Deep Radio Image Segmentation}}
		{Conference Proceedings}
		{\textnormal{\textit{IAU Sypmosium 368 Proceedings}} \\Hattie Stewart, Mark Birkinshaw, Jason Yeung}
\end{entrylist}

%----------------------------------------------------------------------------------------
%	CONFERENCES
%----------------------------------------------------------------------------------------
\vspace{-10 pt}
\cvsect{Conferences}
\begin{entrylist}
    \vspace{-10 pt}
    \entry
		{September 2023}
		{BCRS MIGHTEE Symposium, \textnormal{Bristol}}
		{Talk}
        {}
		% {Radio Image Segmentation in the SKA Era with U-Net}
    \vspace{-10 pt}
    \entry
		{July 2023}
		{National Astronomy Meeting, \textnormal{Cardiff}}
		{Talk}
        {}
		% {Radio Image Segmentation in the SKA Era with U-Net}
    \vspace{-10 pt}
    \entry
		{November 2022}
		{Workshop for Korea-UK AI/ML Research in Fundamental Sciences, \textnormal{Seoul}}
		{Talk}
        {}
		% {Deep Image Segmentation of the Radio Sky}
  %   \entry
		% {August 2022}
		% {IAUS 368: Machine Learning in Astronomy: Possibilities and Pitfalls}
		% {Poster}
		% {Radio image segmentation with variational autoencoders: application to SKA       Data Challenge 1 \\
  %           \textit{presenting author - Mark Birkinshaw}}
    \vspace{-10 pt}
    \entry
		{July 2022}
		{National Astronomy Meeting, \textnormal{Warwick}}
		{Poster}
        {}
		% {Radio Image Segmentation with Variational Autoencoders}
    \vspace{-10 pt}
	\entry
		{June 2022}
		{Machine Learning for Astrophysics, \textnormal{Catania}}
		{Poster and Flash Talk}
        {}
		% {Radio Image Segmentation with Autoencoders}
    \vspace{-10 pt}
    \entry
		{June 2022}
		{AIMLAC Artificial Intelligence Conference, \textnormal{Cardiff}}
		{Talk}
        {}
		% {Radio Image Segmentation with Autoencoders}
\end{entrylist}


%----------------------------------------------------------------------------------------
%	TEACHING EXPERIENCE
%----------------------------------------------------------------------------------------
\vspace{-10pt}
\cvsect{Teaching Experience}
\begin{entrylist}
    \entry
            {2020 - 2022}
            {Teaching Assistant}
            {University of Bristol}
            {Introductory Scientific Computing, Cosmology, Introduction to Computational Physics, Computational Physics}
\end{entrylist}

%----------------------------------------------------------------------------------------
%	ADMINISTRATIVE EXPERIENCE
%----------------------------------------------------------------------------------------
\vspace{-15 pt}
\cvsect{Adminstrative Experience}
\begin{entrylist}
    \entry
            {September 2023}
            {Local Organising Committee - \textnormal{BCRS MIGHTEE Symposium}}
            {University of Bristol}
            {}
\end{entrylist}

%----------------------------------------------------------------------------------------
%	PUBLIC ENGAGEMENT AND OUTREACH
%----------------------------------------------------------------------------------------
\vspace{-10 pt}
\cvsect{Public Engagement \& Outreach}
\begin{entrylist}
    \entry
		{2020 - 2023}
		{Co-Founder and Programme Manager}
		{DataAid}
		{A student led initiative that provides voluntary data science consultancy to a group of selected charities ({\href{https://cdt-aimlac.org/cdt-data-aid.html}{see website}}). I co-founded the programme and have co-ordinated 3 events (1 virtual and 2 hybrid).}
    \entry
            {July 2021}
            {Written in the Stars}
            {Aersopace Bristol}
            {Free Zoom event aimed at Year 9 to Year 11 students, gave a short talk on my research.}
    \entry
            {June 2021}
            {Hareclive in Space - Workshop}
            {We The Curious}
            {Workshop at a local primary school (Hareclive Academy), engaging with student's questions and ideas about space.}
      \entry
		{2015 - 2017}
		{Public Events Coordinator}
		{University of Manchester Physics Outreach (UMPO)}
		{UMPO is a student run organisation that aims to encourage young people into      science through interactive workshops. As Public Events Coordinator, my main    role was to co-manage the organisation of a Physics outreach show.}
\end{entrylist}

%----------------------------------------------------------------------------------------
%	CERTIFICATIONS, SCHOLARSHIPS & TRAINING
%----------------------------------------------------------------------------------------
\vspace{-10 pt}
	\cvsect{Certifications, Scholarships \& Training}
    \vspace{-6pt}
    
    \hspace{26mm} \textbf{Introduction to .NET Programming} - Java Consult \\ 
    
    \hspace{26mm} \textbf{Undergraduate Science Scholarship} - The Ogden Trust

% \fancyfoot[EF]{hello world}
%----------------------------------------------------------------------------------------


% %----------------------------------------------------------------------------------------
% %	LANGUAGES
% %----------------------------------------------------------------------------------------
% \vspace{-10 pt}
% 	\cvsect{Languages}
%     \vspace{-6pt}
    
%     \hspace{26mm} \textbf{English} - B2+, \textbf{ Spanish} - native

% %----------------------------------------------------------------------------------------
\
% \vfill{}
% {\footnotesize\noindent Le informazioni contenute nel presente “Curriculum vitae et studiorium” sono rese sotto la personale responsabilita’ del sottoscritto, ai sensi degli articoli 47 e 47 del Decreto del Presidente della Repubblica, 28 dicembre 2000, numero 445, e successive modifiche e integrazioni, consapevole della responsabilita’ penale prevista dall’articolo 76 del medesimo Decreto per le ipotesi di falsita’ in atti e/o dichiarazioni mendaci.}
\end{document}
